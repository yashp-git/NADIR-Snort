\section{Threat model}
\label{threat_model}

For the purposes of this project we assume attacks which are launched over a computer network from a remote host. Attacks requiring physical or local access to exploit are outside of the scope of this work. 
We also assume an adversary who may be generally aware that the system is equipped with security countermeasures, but is unaware of the specifics of the NADIR system. We assume a typical attack lifecycle 
consisting of repeated rounds of reconnaissance and exploitation, first to gain access to a network, then to penetrate more deeply into the system once initial access has been established. We consider 
adversaries who can run any available tools to gain information about software and services installed on other network hosts. Once they are aware of a vulnerable service on the target server, they can use 
other available exploits to attack the target through that service. 

%Which this threat model, the NADIR system takes an action at the first place where the attackers scan the running services on the
%server. The NADIR system hides the true information of running services then sends the decoy services' information back to the attackers. Thus, we can detect and catch the attackers who attack the
%decoy services.
